\documentclass[12pt,a4paper]{report}

% ============ PACKAGES ============
\usepackage[utf8]{inputenc}
\usepackage[T1]{fontenc}
\usepackage[french]{babel}
\usepackage{geometry}
\usepackage{graphicx}
\usepackage{float}
\usepackage{booktabs}
\usepackage{array}
\usepackage{longtable}
\usepackage{xcolor}
\usepackage{listings}
\usepackage{fancyhdr}
\usepackage{titlesec}
\usepackage{hyperref}
\usepackage{amsmath}
\usepackage{amssymb}
\usepackage{enumitem}
\usepackage{caption}
\usepackage{tabularx}

% ============ PAGE SETUP ============
\geometry{
    left=2.5cm,
    right=2.5cm,
    top=2.5cm,
    bottom=2.5cm
}

% ============ COLORS ============
\definecolor{codegreen}{rgb}{0,0.6,0}
\definecolor{codegray}{rgb}{0.5,0.5,0.5}
\definecolor{codepurple}{rgb}{0.58,0,0.82}
\definecolor{backcolour}{rgb}{0.95,0.95,0.92}
\definecolor{darkblue}{rgb}{0.0,0.0,0.6}
\definecolor{primarycolor}{RGB}{0,82,147}
\definecolor{secondarycolor}{RGB}{70,130,180}

% ============ CODE LISTING STYLE ============
\lstdefinestyle{javastyle}{
    backgroundcolor=\color{backcolour},
    commentstyle=\color{codegreen},
    keywordstyle=\color{darkblue}\bfseries,
    numberstyle=\tiny\color{codegray},
    stringstyle=\color{codepurple},
    basicstyle=\ttfamily\footnotesize,
    breakatwhitespace=false,
    breaklines=true,
    captionpos=b,
    keepspaces=true,
    numbers=left,
    numbersep=5pt,
    showspaces=false,
    showstringspaces=false,
    showtabs=false,
    tabsize=2,
    language=Java,
    frame=single,
    rulecolor=\color{black}
}

\lstset{style=javastyle}

% ============ HYPERREF SETUP ============
\hypersetup{
    colorlinks=true,
    linkcolor=primarycolor,
    filecolor=magenta,
    urlcolor=secondarycolor,
    pdftitle={Conservatoire Virtuel - Système de Gestion d'École de Musique},
    pdfauthor={Hassen Ben Amor, Dalil Adimi},
}

% ============ HEADER/FOOTER ============
\pagestyle{fancy}
\fancyhf{}
\fancyhead[L]{\leftmark}
\fancyhead[R]{Conservatoire Virtuel}
\fancyfoot[C]{\thepage}
\renewcommand{\headrulewidth}{0.4pt}
\renewcommand{\footrulewidth}{0.4pt}

% ============ CHAPTER FORMATTING ============
\titleformat{\chapter}[display]
{\normalfont\huge\bfseries\color{primarycolor}}
{\chaptertitlename\ \thechapter}{20pt}{\Huge}

\titleformat{\section}
{\normalfont\Large\bfseries\color{primarycolor}}
{\thesection}{1em}{}

\titleformat{\subsection}
{\normalfont\large\bfseries\color{secondarycolor}}
{\thesubsection}{1em}{}

% ============ DOCUMENT START ============
\begin{document}

% ============ TITLE PAGE ============
\begin{titlepage}
    \centering
    \vspace*{2cm}
    
    \textsc{\LARGE Université [Nom de l'Université]}\\[0.5cm]
    \textsc{\Large Département d'Informatique}\\[2cm]
    
    \rule{\linewidth}{0.5mm}\\[0.4cm]
    {\huge\bfseries\color{primarycolor} Conservatoire Virtuel}\\[0.2cm]
    {\Large Système de Gestion d'École de Musique}\\[0.2cm]
    \rule{\linewidth}{0.5mm}\\[1.5cm]
    
    \Large\textbf{Projet de Programmation Orientée Objet}\\[2cm]
    
    \begin{minipage}{0.4\textwidth}
        \begin{flushleft}
            \large\textbf{Réalisé par :}\\[0.3cm]
            Hassen \textsc{Ben Amor}\\
            Dalil \textsc{Adimi}
        \end{flushleft}
    \end{minipage}
    \begin{minipage}{0.4\textwidth}
        \begin{flushright}
            \large\textbf{Encadré par :}\\[0.3cm]
            Prof. [Nom du Superviseur]\\[0.5cm]
            \textbf{Année Universitaire :}\\
            2024 -- 2025
        \end{flushright}
    \end{minipage}\\[3cm]
    
    \vfill
    {\large \today}
    
\end{titlepage}

% ============ RÉSUMÉ ============
\chapter*{Résumé}
\addcontentsline{toc}{chapter}{Résumé}

Le projet \textbf{Conservatoire Virtuel} est un système complet de gestion d'école de musique développé en Java. Cette application répond aux besoins opérationnels d'une école de musique privée, offrant des fonctionnalités pour la gestion des étudiants, des professeurs, des forfaits de cours, de la planification des leçons, des paiements et des examens officiels.

Le système a été conçu avec un fort accent sur les principes de programmation orientée objet, comprenant des classes abstraites, des interfaces, le polymorphisme et des constructeurs de copie. L'architecture suit les meilleures pratiques en conception logicielle, incluant la séparation des préoccupations, l'encapsulation et l'utilisation appropriée des modèles de conception.

Les fonctionnalités clés incluent un système de planification robuste avec prévention des conflits, un système de facturation flexible supportant différents types de services, et un module de gestion des examens avec contrôle de capacité et suivi des résultats.

\vspace{1cm}
\textbf{Mots-clés :} Java, POO, École de Musique, Système de Gestion, Planification, Facturation, Classes Abstraites, Interfaces, Polymorphisme

% ============ TABLE DES MATIÈRES ============
\tableofcontents
\listoffigures
\listoftables

% ============ CHAPITRE 1 : INTRODUCTION ============
\chapter{Introduction}

\section{Contexte du Projet}

Une école de musique privée appelée \textbf{Conservatoire Virtuel} nécessite un système d'information pour supporter ses activités internes. L'objectif est de concevoir et implémenter une application qui gère tous les aspects des opérations de l'école de manière efficace et professionnelle.

\section{Objectifs du Projet}

Les principaux objectifs de ce projet sont :

\begin{enumerate}[label=\arabic*.]
    \item Concevoir un modèle de domaine complet pour une école de musique
    \item Implémenter des concepts avancés de programmation orientée objet
    \item Développer une application console fonctionnelle
    \item Démontrer les bonnes pratiques d'ingénierie logicielle
    \item Créer une documentation professionnelle incluant des diagrammes UML
\end{enumerate}

\section{Périmètre}

Le système gère les aspects suivants :

\begin{itemize}
    \item \textbf{Gestion des Personnes :} Étudiants et professeurs avec leurs attributs et relations
    \item \textbf{Gestion des Services :} Forfaits de cours, leçons individuelles et locations d'instruments
    \item \textbf{Planification :} Planification des leçons avec prévention des conflits et réservation de ressources
    \item \textbf{Gestion Financière :} Paiements, facturation et comptabilité
    \item \textbf{Gestion des Examens :} Examens officiels, inscription et résultats
\end{itemize}

\section{Structure du Document}

Ce rapport est organisé comme suit :

\begin{itemize}
    \item \textbf{Chapitre 2 :} Analyse des Besoins
    \item \textbf{Chapitre 3 :} Conception du Système avec UML
    \item \textbf{Chapitre 4 :} Détails d'Implémentation
    \item \textbf{Chapitre 5 :} Discussion des Concepts POO
    \item \textbf{Chapitre 6 :} Tests et Résultats
    \item \textbf{Chapitre 7 :} Conclusion
\end{itemize}

% ============ CHAPITRE 2 : BESOINS ============
\chapter{Analyse des Besoins}

\section{Besoins Fonctionnels}

\subsection{Gestion des Étudiants}

Un étudiant est identifié par un ID unique avec les attributs suivants :
\begin{itemize}
    \item Nom, prénom
    \item Adresse, date de naissance, téléphone, email
    \item Niveau (Débutant/Intermédiaire/Avancé)
    \item Instruments préférés
\end{itemize}

Les étudiants peuvent s'inscrire à :
\begin{itemize}
    \item Des Forfaits de Cours (N leçons avec dates de validité)
    \item Des leçons individuelles (facturées par leçon)
\end{itemize}

Le système suit :
\begin{itemize}
    \item Les heures achetées
    \item Les heures restantes
    \item L'historique d'utilisation
\end{itemize}

\subsection{Gestion des Professeurs}

Chaque professeur possède :
\begin{itemize}
    \item ID, nom, qualifications
    \item Spécialisations (instruments qu'ils peuvent enseigner)
    \item Tarif horaire
    \item Planning de disponibilité
\end{itemize}

\subsection{Forfaits et Services}

L'école propose :
\begin{itemize}
    \item Forfaits musique (heures fixes)
    \item Forfaits illimités
    \item Leçons de groupe ou individuelles
    \item Leçons payantes uniques
    \item Location d'instruments
    \item Réservation de salles
\end{itemize}

\subsection{Système de Planification}

Le système permet de planifier :
\begin{itemize}
    \item Leçons (professeur + étudiant + salle + instrument)
    \item Locations de salles
    \item Sessions d'examen
\end{itemize}

Pour chaque activité planifiée :
\begin{itemize}
    \item Date et heure de début
    \item Durée
    \item Ressources assignées
    \item Statut (planifié/terminé/annulé)
\end{itemize}

\textbf{Important :} Les conflits de planification doivent être évités.

\subsection{Gestion des Examens}

L'école organise des examens officiels :
\begin{itemize}
    \item Les examens ont un nom, un instrument, une date et une capacité
    \item Les étudiants peuvent s'inscrire
    \item Les résultats sont enregistrés : réussi/échoué et score optionnel
\end{itemize}

\subsection{Paiements et Facturation}

Le système suit :
\begin{itemize}
    \item Les paiements
    \item Les soldes impayés
    \item Les prix des forfaits, locations et leçons
\end{itemize}

\section{Règles Métier}

\begin{enumerate}
    \item Les heures de forfait expirent à la date de fin du forfait
    \item Si une leçon est annulée moins de 24 heures avant, elle est comptée comme consommée
    \item L'inscription aux examens est fermée lorsque la capacité maximale est atteinte
    \item Une leçon de groupe consomme une heure de chaque participant
\end{enumerate}

% ============ CHAPITRE 3 : CONCEPTION ============
\chapter{Conception du Système}

\section{Vue d'Ensemble de l'Architecture}

Le système suit une architecture en couches :

\begin{enumerate}
    \item \textbf{Couche Présentation :} Interface utilisateur console (\texttt{ConservatoireApp})
    \item \textbf{Couche Service :} Logique métier (\texttt{SchedulingService}, \texttt{PaymentService}, \texttt{ExamService})
    \item \textbf{Couche Repository :} Stockage des données (\texttt{DataRepository})
    \item \textbf{Couche Modèle :} Entités du domaine (Person, Service, ScheduledActivity, etc.)
\end{enumerate}

\section{Structure des Packages}

\begin{figure}[H]
\centering
\begin{verbatim}
com.music.school/
├── model/
│   ├── person/      (Student, Teacher)
│   ├── service/     (CoursePackage, IndividualLesson, InstrumentRental)
│   ├── scheduling/  (Lesson, RoomBooking)
│   ├── resource/    (Room, Instrument)
│   ├── exam/        (Exam, ExamRegistration)
│   └── billing/     (Invoice, Payment)
├── interfaces/      (Schedulable, Billable, Bookable)
├── enums/           (Level, ActivityStatus, PaymentStatus, etc.)
├── service/         (Services métier)
├── repository/      (Stockage de données)
└── data/            (Initialisation des données de test)
\end{verbatim}
\caption{Structure des Packages du Projet}
\label{fig:packages}
\end{figure}

\section{Diagramme de Classes}

\subsection{Hiérarchie Person}

\begin{table}[H]
\centering
\caption{Hiérarchie de la Classe Person}
\label{tab:person-hierarchy}
\begin{tabular}{|l|l|l|p{5cm}|}
\hline
\textbf{Classe} & \textbf{Type} & \textbf{Parent} & \textbf{Attributs Clés} \\
\hline
Person & Abstraite & - & id, firstName, lastName, email, dateOfBirth \\
\hline
Student & Concrète & Person & level, preferredInstruments, packageHours \\
\hline
Teacher & Concrète & Person & hourlyRate, specializations, availability \\
\hline
\end{tabular}
\end{table}

\begin{figure}[H]
\centering
\begin{verbatim}
                    +------------------+
                    |   <<abstract>>   |
                    |      Person      |
                    +------------------+
                    | # id: String     |
                    | # firstName      |
                    | # lastName       |
                    | # email          |
                    +------------------+
                    | + getRole()      |
                    | + copy()         |
                    +------------------+
                           /\
                          /  \
                         /    \
            +-----------+      +-----------+
            |  Student  |      |  Teacher  |
            +-----------+      +-----------+
            | - level   |      | - hourlyRate|
            | - hours   |      | - specs   |
            +-----------+      +-----------+
            |+consumeHrs|      |+canTeach()|
            +-----------+      +-----------+
                                    |
                                    | implements
                                    v
                            <<Bookable>>
\end{verbatim}
\caption{Diagramme de Classes - Hiérarchie Person}
\label{fig:class-person}
\end{figure}

\subsection{Hiérarchie Service}

\begin{table}[H]
\centering
\caption{Hiérarchie de la Classe Service}
\label{tab:service-hierarchy}
\begin{tabular}{|l|l|l|p{5cm}|}
\hline
\textbf{Classe} & \textbf{Type} & \textbf{Interface} & \textbf{Attributs Clés} \\
\hline
Service & Abstraite & Billable & id, name, price, studentId, paid \\
\hline
CoursePackage & Concrète & Billable & totalHours, usedHours, instrument \\
\hline
IndividualLesson & Concrète & Billable & instrument, durationMinutes \\
\hline
InstrumentRental & Concrète & Billable & dailyRate, depositAmount \\
\hline
\end{tabular}
\end{table}

\begin{figure}[H]
\centering
\begin{verbatim}
     <<interface>>                  +------------------+
       Billable                     |   <<abstract>>   |
    +-------------+                 |     Service      |
    |+calculateAmt| <-------------- +------------------+
    |+isPaid()    |   implements    | # id: String     |
    +-------------+                 | # price          |
                                    | # studentId      |
                                    +------------------+
                                    | + isValid()      |
                                    +------------------+
                                           /|\
                        ___________________/|\___________________
                       /                    |                    \
          +---------------+      +----------------+      +------------------+
          | CoursePackage |      |IndividualLesson|      | InstrumentRental |
          +---------------+      +----------------+      +------------------+
          | - totalHours  |      | - instrument   |      | - dailyRate      |
          | - usedHours   |      | - duration     |      | - deposit        |
          +---------------+      +----------------+      +------------------+
\end{verbatim}
\caption{Diagramme de Classes - Hiérarchie Service}
\label{fig:class-service}
\end{figure}

\subsection{Hiérarchie ScheduledActivity}

\begin{table}[H]
\centering
\caption{Hiérarchie de la Classe ScheduledActivity}
\label{tab:activity-hierarchy}
\begin{tabular}{|l|l|l|p{5cm}|}
\hline
\textbf{Classe} & \textbf{Type} & \textbf{Interface} & \textbf{Attributs Clés} \\
\hline
ScheduledActivity & Abstraite & Schedulable & id, dateTime, duration, status, roomId \\
\hline
Lesson & Concrète & Schedulable & teacherId, studentIds, instrument \\
\hline
RoomBooking & Concrète & Schedulable & studentId, hourlyRate, purpose \\
\hline
\end{tabular}
\end{table}

\subsection{Résumé Complet des Classes}

\begin{table}[H]
\centering
\caption{Résumé de Toutes les Classes}
\label{tab:all-classes}
\begin{tabular}{|l|l|l|l|}
\hline
\textbf{Classe} & \textbf{Type} & \textbf{Parent} & \textbf{Interfaces} \\
\hline
Person & Abstraite & - & - \\
Student & Concrète & Person & - \\
Teacher & Concrète & Person & Bookable \\
\hline
Service & Abstraite & - & Billable \\
CoursePackage & Concrète & Service & Billable \\
IndividualLesson & Concrète & Service & Billable \\
InstrumentRental & Concrète & Service & Billable \\
\hline
ScheduledActivity & Abstraite & - & Schedulable \\
Lesson & Concrète & ScheduledActivity & Schedulable \\
RoomBooking & Concrète & ScheduledActivity & Schedulable \\
\hline
Room & Concrète & - & Bookable \\
Instrument & Concrète & - & Bookable \\
Exam & Concrète & - & - \\
Invoice & Concrète & - & - \\
Payment & Concrète & - & - \\
\hline
\end{tabular}
\end{table}

\section{Diagramme de Séquence : Planification d'une Leçon}

Le processus de planification d'une leçon suit ces étapes :

\begin{table}[H]
\centering
\caption{Diagramme de Séquence - Étapes de Planification}
\label{tab:sequence}
\begin{tabular}{|c|l|l|l|}
\hline
\textbf{Étape} & \textbf{De} & \textbf{Vers} & \textbf{Message} \\
\hline
1 & Utilisateur & SchedulingService & scheduleLesson() \\
2 & SchedulingService & DataRepository & getTeacher(id) \\
3 & SchedulingService & Teacher & canTeach(instrument) \\
4 & SchedulingService & DataRepository & checkConflicts() \\
5 & SchedulingService & Teacher & isAvailableAt() \\
6 & SchedulingService & Room & isAvailableAt() \\
7 & SchedulingService & Teacher & addBooking() \\
8 & SchedulingService & Room & addBooking() \\
9 & SchedulingService & DataRepository & addScheduledActivity() \\
10 & SchedulingService & Utilisateur & return Lesson \\
\hline
\end{tabular}
\end{table}

\begin{figure}[H]
\centering
\begin{verbatim}
User          SchedulingService     DataRepository      Teacher         Room
  |                  |                    |                |              |
  |--scheduleLesson->|                    |                |              |
  |                  |---getTeacher(id)-->|                |              |
  |                  |<------teacher------|                |              |
  |                  |                    |                |              |
  |                  |--------canTeach(instrument)-------->|              |
  |                  |<--------------true------------------|              |
  |                  |                    |                |              |
  |                  |---checkConflicts-->|                |              |
  |                  |<---no conflicts----|                |              |
  |                  |                    |                |              |
  |                  |--------isAvailableAt()------------->|              |
  |                  |<--------------true------------------|              |
  |                  |                    |                |              |
  |                  |-----------------isAvailableAt()-------------------->|
  |                  |<--------------------true----------------------------|
  |                  |                    |                |              |
  |                  |--------addBooking()---------------->|              |
  |                  |-----------------addBooking()----------------------->|
  |                  |                    |                |              |
  |                  |--addScheduledAct-->|                |              |
  |                  |                    |                |              |
  |<----Lesson-------|                    |                |              |
\end{verbatim}
\caption{Diagramme de Séquence - Planification d'une Leçon}
\label{fig:sequence}
\end{figure}

\section{Diagramme d'Activité : Inscription à un Examen}

\begin{figure}[H]
\centering
\begin{verbatim}
                          ( Début )
                              |
                              v
                    +-------------------+
                    | Sélectionner Exam |
                    +-------------------+
                              |
                              v
                    <Inscription ouverte?>
                         /         \
                       Non          Oui
                        |            |
                        v            v
                   [Erreur]   <Avant deadline?>
                        ^        /         \
                        |      Non          Oui
                        |       |            |
                        +-------+            v
                        ^          <Places disponibles?>
                        |              /         \
                        |            Non          Oui
                        |             |            |
                        +-------------+            v
                        ^                <Déjà inscrit?>
                        |                  /         \
                        |                Oui          Non
                        |                 |            |
                        +-----------------+            v
                                          +-------------------+
                                          | Créer Inscription |
                                          +-------------------+
                                                    |
                                                    v
                                          +-------------------+
                                          | Retourner Succès  |
                                          +-------------------+
                                                    |
                                                    v
                                                ( Fin )
\end{verbatim}
\caption{Diagramme d'Activité - Processus d'Inscription à un Examen}
\label{fig:activity}
\end{figure}

% ============ CHAPITRE 4 : IMPLÉMENTATION ============
\chapter{Détails d'Implémentation}

\section{Technologies Utilisées}

\begin{itemize}
    \item \textbf{Langage :} Java 17
    \item \textbf{Outil de Build :} Maven 3.x
    \item \textbf{IDE :} IntelliJ IDEA / VS Code
\end{itemize}

\section{Détails Clés d'Implémentation}

\subsection{Classe Abstraite Person}

\begin{lstlisting}[caption={Classe Abstraite Person}]
public abstract class Person {
    protected String id;
    protected String firstName;
    protected String lastName;
    protected String email;
    protected LocalDate dateOfBirth;
    
    // Constructeur de copie
    protected Person(Person other) {
        this.id = other.id;
        this.firstName = other.firstName;
        this.lastName = other.lastName;
        this.email = other.email;
        this.dateOfBirth = other.dateOfBirth;
    }
    
    // Methodes abstraites
    protected abstract String getIdPrefix();
    public abstract String getRole();
    public abstract Person copy();
    
    public String getFullName() {
        return firstName + " " + lastName;
    }
}
\end{lstlisting}

\subsection{Interface Schedulable}

\begin{lstlisting}[caption={Interface Schedulable}]
public interface Schedulable {
    LocalDateTime getScheduledDateTime();
    Duration getDuration();
    ActivityStatus getStatus();
    
    default LocalDateTime getEndDateTime() {
        return getScheduledDateTime().plus(getDuration());
    }
    
    default boolean conflictsWith(Schedulable other) {
        LocalDateTime thisStart = this.getScheduledDateTime();
        LocalDateTime thisEnd = this.getEndDateTime();
        LocalDateTime otherStart = other.getScheduledDateTime();
        LocalDateTime otherEnd = other.getEndDateTime();
        
        return thisStart.isBefore(otherEnd) && 
               thisEnd.isAfter(otherStart);
    }
    
    default boolean canCancelWithoutPenalty() {
        return LocalDateTime.now().plusHours(24)
            .isBefore(getScheduledDateTime());
    }
}
\end{lstlisting}

\subsection{Détection des Conflits}

\begin{lstlisting}[caption={Détection des Conflits de Planification}]
public List<String> checkSchedulingConflicts(
        String teacherId, List<String> studentIds,
        String roomId, LocalDateTime dateTime, 
        int durationMinutes) {
    
    List<String> conflicts = new ArrayList<>();
    
    // Verifier les conflits du professeur
    for (Lesson lesson : repository.getTeacherLessons(teacherId)) {
        if (tempSchedulable.conflictsWith(lesson)) {
            conflicts.add("Le professeur a une autre lecon");
        }
    }
    
    // Verifier les conflits de salle
    for (ScheduledActivity activity : 
            repository.getRoomActivities(roomId)) {
        if (tempSchedulable.conflictsWith(activity)) {
            conflicts.add("La salle est deja reservee");
        }
    }
    
    return conflicts;
}
\end{lstlisting}

\subsection{Exemple de Constructeur de Copie}

\begin{lstlisting}[caption={Constructeur de Copie de Student}]
public class Student extends Person {
    private Level level;
    private List<String> preferredInstruments;
    private Map<String, Integer> packageHours;
    
    // Constructeur de copie
    public Student(Student other) {
        super(other);  // Appel au constructeur parent
        this.level = other.level;
        // Copie profonde des collections
        this.preferredInstruments = 
            new ArrayList<>(other.preferredInstruments);
        this.packageHours = 
            new HashMap<>(other.packageHours);
    }
    
    @Override
    public Person copy() {
        return new Student(this);
    }
}
\end{lstlisting}

% ============ CHAPITRE 5 : CONCEPTS POO ============
\chapter{Concepts de Programmation Orientée Objet}

\section{Classes Abstraites}

Le système implémente trois classes abstraites :

\begin{table}[H]
\centering
\caption{Résumé des Classes Abstraites}
\label{tab:abstract}
\begin{tabular}{|l|l|p{5cm}|}
\hline
\textbf{Classe Abstraite} & \textbf{Étendue Par} & \textbf{Méthodes Abstraites} \\
\hline
Person & Student, Teacher & getIdPrefix(), getRole(), copy() \\
\hline
Service & CoursePackage, IndividualLesson, InstrumentRental & getIdPrefix(), isValid(), copy() \\
\hline
ScheduledActivity & Lesson, RoomBooking & getIdPrefix(), getActivityType(), consumesLessonHours() \\
\hline
\end{tabular}
\end{table}

\section{Interfaces}

Trois interfaces définissent des capacités indépendantes :

\begin{table}[H]
\centering
\caption{Résumé des Interfaces}
\label{tab:interfaces}
\begin{tabular}{|l|l|p{4cm}|}
\hline
\textbf{Interface} & \textbf{Implémentée Par} & \textbf{Méthodes Clés} \\
\hline
Schedulable & ScheduledActivity, Lesson, RoomBooking & getScheduledDateTime(), conflictsWith() \\
\hline
Billable & Service et sous-classes & calculateAmount(), getBillingDescription() \\
\hline
Bookable & Teacher, Room, Instrument & isAvailableAt(), addBooking() \\
\hline
\end{tabular}
\end{table}

\section{Polymorphisme}

\subsection{Collections Polymorphiques}

\begin{lstlisting}[caption={Collections Polymorphiques}]
// Liste de Person contenant Student et Teacher
List<Person> people = new ArrayList<>();
people.addAll(repository.getAllStudents());
people.addAll(repository.getAllTeachers());

for (Person person : people) {
    // Dispatch dynamique - sortie differente pour chaque type
    System.out.println(person.getRole());
}
\end{lstlisting}

\subsection{Appels de Méthodes Polymorphiques}

\begin{lstlisting}[caption={Facturation Polymorphique}]
// Tous les services implementent Billable
List<Billable> billables = new ArrayList<>();
billables.addAll(packages);     // CoursePackage
billables.addAll(lessons);      // IndividualLesson
billables.addAll(rentals);      // InstrumentRental

for (Billable b : billables) {
    // calculateAmount() varie selon le type
    total = total.add(b.calculateAmount());
}
\end{lstlisting}

\section{Constructeurs de Copie}

Les constructeurs de copie sont implémentés dans toutes les classes principales :

\begin{itemize}
    \item Student, Teacher
    \item CoursePackage, IndividualLesson, InstrumentRental
    \item Lesson, RoomBooking
    \item Room, Instrument
    \item Exam, Invoice, Payment
\end{itemize}

\section{Encapsulation}

\begin{itemize}
    \item Tous les champs sont \texttt{private} ou \texttt{protected}
    \item Les getters retournent des copies défensives des collections mutables
    \item Les setters valident les données d'entrée
\end{itemize}

\begin{lstlisting}[caption={Copie Défensive}]
public List<String> getPreferredInstruments() {
    return new ArrayList<>(preferredInstruments);
}
\end{lstlisting}

% ============ CHAPITRE 6 : TESTS ============
\chapter{Tests et Résultats}

\section{Données de Test}

Le système inclut des données de test complètes :

\begin{itemize}
    \item \textbf{8 Étudiants} (dont 3 mineurs)
    \item \textbf{5 Professeurs} (Piano, Violon, Guitare, Batterie, Flûte)
    \item \textbf{8 Salles} (Studios, Salles de pratique, Salle d'ensemble)
    \item \textbf{7 Instruments} à louer
    \item \textbf{5 Forfaits de Cours}
    \item \textbf{4 Examens à Venir}
\end{itemize}

\section{Captures d'Écran de l'Application}

\subsection{Menu Principal}

\begin{figure}[H]
\centering
\begin{verbatim}
+============================================+
|              MENU PRINCIPAL                |
+============================================+
|  1. Gerer Etudiants et Professeurs         |
|  2. Gerer Forfaits et Lecons               |
|  3. Gerer Planification et Reservations    |
|  4. Gerer Paiements et Facturation         |
|  5. Gerer Examens et Resultats             |
|  6. Demonstrer Concepts POO                |
|  0. Quitter                                |
+============================================+
\end{verbatim}
\caption{Menu Principal de l'Application}
\label{fig:menu}
\end{figure}

\subsection{Liste des Étudiants}

\begin{figure}[H]
\centering
\begin{verbatim}
=== TOUS LES ETUDIANTS (8) ===
ID              Nom                  Niveau         Heures
--------------------------------------------------------------------
STU-A1B2C3D4    Alice Moreau        Intermediaire   8
STU-E5F6G7H8    Thomas Leroy        Debutant        6
STU-I9J0K1L2    Emma Dubois         Avance          Illimite
\end{verbatim}
\caption{Affichage de la Liste des Étudiants}
\label{fig:students}
\end{figure}

\subsection{Détection des Conflits}

\begin{figure}[H]
\centering
\begin{verbatim}
X Erreur: Conflits de planification detectes:
  - Le professeur a une autre lecon a 2025-01-15T10:00
  - La salle est deja reservee a 2025-01-15T10:00
\end{verbatim}
\caption{Message d'Erreur de Conflit}
\label{fig:conflict}
\end{figure}

% ============ CHAPITRE 7 : CONCLUSION ============
\chapter{Conclusion}

\section{Résumé}

Le projet \textbf{Conservatoire Virtuel} implémente avec succès un système complet de gestion d'école de musique répondant à toutes les exigences spécifiées.

\section{Checklist des Exigences POO}

\begin{table}[H]
\centering
\caption{Exigences POO Satisfaites}
\label{tab:checklist}
\begin{tabular}{|l|c|l|}
\hline
\textbf{Exigence} & \textbf{Statut} & \textbf{Implémentation} \\
\hline
Classes Abstraites (2+) & $\checkmark$ & Person, Service, ScheduledActivity \\
Interfaces (2+) & $\checkmark$ & Schedulable, Billable, Bookable \\
Polymorphisme & $\checkmark$ & Collections et appels de méthodes \\
Constructeurs de Copie & $\checkmark$ & Toutes les classes principales \\
Héritage & $\checkmark$ & Multiples hiérarchies \\
Composition & $\checkmark$ & Teacher-TimeRange, Exam-Registration \\
Encapsulation & $\checkmark$ & Champs privés, copies défensives \\
\hline
\end{tabular}
\end{table}

\section{Améliorations Futures}

\begin{itemize}
    \item Intégration base de données
    \item Interface graphique utilisateur
    \item API REST
    \item Notifications par email
\end{itemize}

% ============ ANNEXE ============
\appendix

\chapter{Comment Exécuter}

\begin{lstlisting}[language=bash, caption={Commandes d'Exécution}]
# Naviguer vers le repertoire du projet
cd java-project

# Compiler et executer avec Maven
mvn clean compile
mvn exec:java -Dexec.mainClass="com.music.school.ConservatoireApp"

# Ou packager et executer comme JAR
mvn package
java -jar target/conservatoire-virtuel-1.0.0.jar
\end{lstlisting}

% ============ BIBLIOGRAPHIE ============
\begin{thebibliography}{9}

\bibitem{java17}
Oracle Corporation,
\textit{Java SE 17 Documentation},
2021.

\bibitem{oop}
Bertrand Meyer,
\textit{Object-Oriented Software Construction},
Prentice Hall, 2nd Edition, 1997.

\bibitem{cleancode}
Robert C. Martin,
\textit{Clean Code: A Handbook of Agile Software Craftsmanship},
Prentice Hall, 2008.

\end{thebibliography}

\end{document}

